%%%%%%%%%%%%%%%%%%%%%%%%%%%%%%%%%%%%%%%%%%%%%%%%%%%%%%%%%%%%%%%%%%%%%%%%%%%%%%%
%%%%%%%%%%%%%%%%%%%%%%%%%%%%%%%%%%%%%%%%%%%%%%%%%%%%%%%%%%%%%%%%%%%%%%%%%%%%%%%
%%%%%%%%%%%%%%%%%%%%%%%%%%%%%%%%%%%%%%%%%%%%%%%%%%%%%%%%%%%%%%%%%%%%%%%%%%%%%%%

%% definuji třídu a vlastnosti dokumentu jako celku ---------------------------
\documentclass[t]{beamer}


%% ----------------------------------------------------------------------------
%% nastavuji titulní stranu a globální parametry ------------------------------
%% ----------------------------------------------------------------------------

\mode<presentation>{

%% vkládám své vlastní styly, knihovny, některé globální parametry ------------
%%%%%%%%%%%%%%%%%%%%%%%%%%%%%%%%%%%%%%%%%%%%%%%%%%%%%%%%%%%%%%%%%%%%%%%%%%%%%%%
%%%%%%%%%%%%%%%%%%%%%%%%%%%%%%%%%%%%%%%%%%%%%%%%%%%%%%%%%%%%%%%%%%%%%%%%%%%%%%%
%%%%%%%%%%%%%%%%%%%%%%%%%%%%%%%%%%%%%%%%%%%%%%%%%%%%%%%%%%%%%%%%%%%%%%%%%%%%%%%

% téma prezentace -------------------------------------------------------------
\usetheme{Frankfurt}

% článková matematická notace -------------------------------------------------
\usefonttheme[onlymath]{serif}

% loaduju některé balíčky -----------------------------------------------------
\usepackage[T1]{fontenc}
\usepackage[utf8]{inputenc}
\usepackage[english, czech]{babel}
\usepackage{csquotes}
\usepackage{amsmath}
\usepackage{physics}
\usepackage{amsfonts}
\usepackage{array}
\usepackage{multirow}
\usepackage{graphicx}
\usepackage{blindtext}
\usepackage{scrextend}
\usepackage{remreset}
\usepackage{hyperref}
\usepackage{listings}


% české uvozovky --------------------------------------------------------------
\DeclareQuoteAlias{german}{czech}
\MakeOuterQuote{"}

% změna velikosti fontu -------------------------------------------------------
\newcommand\reduceFontSize{\fontsize{8}{10}\selectfont}

% definuji rozdílný footer na titulní straně a dalších slidech ----------------
\defbeamertemplate*{footline}{my footline}
{
    \ifnum \insertpagenumber=1
        \leavevmode%
        \hbox{%
            \pgfsetfillopacity{1}\begin{beamercolorbox}[wd=.23\paperwidth,
              ht=2.25ex,dp=1ex,center]{author in head/foot}%
              \usebeamerfont{author in head/foot}\pgfsetfillopacity{1}
            \end{beamercolorbox}%
            \pgfsetfillopacity{1}\begin{beamercolorbox}[wd=.50\paperwidth,
              ht=2.25ex,dp=1ex,center]{title in head/foot}%
              \usebeamerfont{title in head/foot}\pgfsetfillopacity{1}
            \end{beamercolorbox}%
            \pgfsetfillopacity{1}\begin{beamercolorbox}[wd=.27\paperwidth,
              ht=2.25ex,dp=1ex,right]{date in head/foot}%
              \insertframenumber{} / \inserttotalframenumber\hspace*{2ex}
            \end{beamercolorbox}
          }%
        \vskip0pt%
    \else
        \leavevmode%
        \hbox{%
            \pgfsetfillopacity{1}\begin{beamercolorbox}[wd=.23\paperwidth,
              ht=2.25ex,dp=1ex,center]{author in head/foot}%
              \usebeamerfont{author in head/foot}\pgfsetfillopacity{1}%
                \insertshortauthor
            \end{beamercolorbox}%
            \pgfsetfillopacity{1}\begin{beamercolorbox}[wd=.50\paperwidth,
              ht=2.25ex,dp=1ex,center]{title in head/foot}%
              \usebeamerfont{title in head/foot}\pgfsetfillopacity{1}%
                \insertshorttitle
            \end{beamercolorbox}%
            \pgfsetfillopacity{1}\begin{beamercolorbox}[wd=.16\paperwidth,
              ht=2.25ex,dp=1ex,center]{date in head/foot}%
              \usebeamerfont{date in head/foot}\pgfsetfillopacity{1}%
                \insertshortdate{}
            \end{beamercolorbox}%
            \pgfsetfillopacity{1}\begin{beamercolorbox}[wd=.11\paperwidth,
              ht=2.25ex,dp=1ex,right]{date in head/foot}%  
              \insertframenumber{} / \inserttotalframenumber\hspace*{2ex}
            \end{beamercolorbox}
          }%
        \vskip0pt%
    \fi
}

\setbeamertemplate{footline}[my footline]


% počítadlo stránek v headeru -------------------------------------------------
\makeatletter
    \@removefromreset{subsection}{section}
\makeatother
\setcounter{subsection}{1}

% nastavuji highlighting chunků kódu na jazyk R a další detaily ---------------
%\captionsetup[lstlisting]{position = bottom}
\definecolor{my_gray}{rgb}{0.4,0.4,0.4}
\renewcommand{\lstlistingname}{Kód}  %% předefinovávám název chunků
\lstdefinestyle{custom_R}{
  basicstyle = \ttfamily\scriptsize,
  belowcaptionskip = 1\baselineskip,
  breaklines = true,
  frame = L,
  xleftmargin = \parindent,
  language = R,
  showstringspaces = false,
  keywordstyle = \bfseries\color{green!40!black},
  commentstyle = \itshape\color{purple!40!black},
  identifierstyle = \color{blue},
  stringstyle = \color{orange},
  numbers = left,
  numbersep = 12pt,
  numberstyle = \small\color{my_gray},
  texcl = true,
  alsoother = {\#},
  inputencoding = utf8,
  extendedchars = true,
  literate = %
  {á}{{\'a}}1
  {č}{{\v{c}}}1
  {ď}{{\v{d}}}1
  {é}{{\'e}}1
  {ě}{{\v{e}}}1
  {í}{{\'i}}1
  {ň}{{\v{n}}}1
  {ó}{{\'o}}1
  {ř}{{\v{r}}}1
  {š}{{\v{s}}}1
  {ť}{{\v{t}}}1
  {ú}{{\'u}}1
  {ů}{{\r{u}}}1
  {ý}{{\'y}}1
  {ž}{{\v{z}}}1
  {Á}{{\'A}}1
  {Č}{{\v{C}}}1
  {Ď}{{\v{D}}}1
  {É}{{\'E}}1
  {Ě}{{\v{E}}}1
  {Í}{{\'I}}1
  {Ň}{{\v{N}}}1
  {Ó}{{\'O}}1
  {Ř}{{\v{R}}}1
  {Š}{{\v{S}}}1
  {Ť}{{\v{T}}}1
  {Ú}{{\'U}}1
  {Ů}{{\r{U}}}1
  {Ý}{{\'Y}}1
  {Ž}{{\v{Z}}}1
}
\lstset{escapechar = @, style = custom_R}


%%%%%%%%%%%%%%%%%%%%%%%%%%%%%%%%%%%%%%%%%%%%%%%%%%%%%%%%%%%%%%%%%%%%%%%%%%%%%%%
%%%%%%%%%%%%%%%%%%%%%%%%%%%%%%%%%%%%%%%%%%%%%%%%%%%%%%%%%%%%%%%%%%%%%%%%%%%%%%%
%%%%%%%%%%%%%%%%%%%%%%%%%%%%%%%%%%%%%%%%%%%%%%%%%%%%%%%%%%%%%%%%%%%%%%%%%%%%%%%







}


%% nastavuji titulní stranu ---------------------------------------------------

\title[Vývoj aplikací v \textsf{R}]{Vývoj aplikací v \textsf{R}}
\subtitle{Statistické dýchánky na VŠE}
\author[Lubomír Štěpánek]{Lubomír Štěpánek}
\institute[FBMI, 1. LF UK]{%
  Katedra biomedicínské informatiky \\
  Fakulta biomedicínského inženýrství \\
  České vysoké učení technické v Praze \\
  --- \\
  Centrum podpory multimediálních forem výuky \\
  Oddělení výpočetní techniky \\
  1. lékařská fakulta \\
  Univerzita Karlova v Praze
}
\date{25. května 2017}


%% ----------------------------------------------------------------------------
%% začínám dokument -----------------------------------------------------------
%% ----------------------------------------------------------------------------

\begin{document}


%% titulní strana -------------------------------------------------------------

{
\setbeamertemplate{headline}{} 
\begin{frame}
  \titlepage
\end{frame}
}


%% ----------------------------------------------------------------------------
%% sekce "rozjezd" ------------------------------------------------------------
%% ----------------------------------------------------------------------------

\section{Rozjezd}

%% obsah ----------------------------------------------------------------------

\begin{frame}
  \frametitle{Obsah}
  \tableofcontents
\end{frame}


%% motivace -------------------------------------------------------------------

\begin{frame}
  \frametitle{Motivace}
  \begin{itemize}
    \item možnost jednoduše vytvořit vlastní spustitelnou či webovou
    aplikaci (jen) pomocí jazyka \textsf{R} a balíčku \texttt{Shiny},
    a to i bez znalostí webařiny
    \item zároveň možnost aplikaci libovolně vylepšit při znalosti
    HTML (\underline{H}yper\underline{T}ext \underline{M}arkup
    \underline{L}anguage), CSS (\underline{C}ascading \underline{S}tyle
    \underline{S}heets) či \texttt{javascriptu}
    \item aplikace může s výhodou využít prakticky libovolnou
    funkcionalitu \textsf{R}
  \end{itemize}
\end{frame}


%% úvod -----------------------------------------------------------------------

\begin{frame}
  \frametitle{Úvod}
  \begin{itemize}
    \item smyslem je napsat kód v \textsf{R}, který by fungoval
    i v konzoli, resp. RStudiu, doplnit prvky uživatelského
    rozhraní, a samostatně spouštět či vystavit online
    \item základním frameworkem je \textsf{R}-kový balíček \texttt{Shiny}
  \end{itemize}
\end{frame}


%% minimal working example ----------------------------------------------------

\begin{frame}
  \frametitle{Minimal Working Example}
  \begin{itemize}
    \item aplikace \texttt{hello\_world} a aplikace \texttt{kalkulator}
    \begin{center}
      \href{https://github.com/LStepanek/Vyvoj\_aplikaci\_v\_R/}{%
        \framebox{https://github.com/LStepanek/Vyvoj\_aplikaci\_v\_R/}%
      }
    \end{center}
  \end{itemize}
\end{frame}


%% ----------------------------------------------------------------------------
%% sekce "první aplikace" -----------------------------------------------------
%% ----------------------------------------------------------------------------

\section{První aplikace}

%% komponenty typické aplikace ------------------------------------------------

\begin{frame}
  \frametitle{Komponenty typické aplikace}
  \begin{itemize}
    \item obligátní
    \begin{itemize}
      \item \texttt{ui.R}
      \item \texttt{server.R}
    \end{itemize}
    \item nadstandardní (nejsou nutné)
    \begin{itemize}
      \item \texttt{global.R}
      \item \texttt{index.html}
      \item libovolné soubory s příponou \texttt{.R}
      \item složka \texttt{www}
      \begin{itemize}
        \item obrázky, CSS třídy, \texttt{javascriptové} funkce atd.
      \end{itemize}
      \item cokoliv dalšího, co \textsf{R}, \texttt{Shiny}
      a web zná a "snese"
    \end{itemize}
  \end{itemize}
\end{frame}


%% ui.R -----------------------------------------------------------------------

\begin{frame}
  \frametitle{\texttt{ui.R}}
  \begin{itemize}
    \item \underline{u}ser \underline{i}nterface (uživatelské rozhraní)
    rozhraní, a samostatně spouštět či vystavit online
    \item skript s \textsf{R}-kovým kódem určující, které prvky a jak budou
    uživateli zobrazeny a případně je bude moci měnit (vstupy),
    eventuálně získat (výstupy)
  \end{itemize}
\end{frame}


%% typické ui.R ---------------------------------------------------------------

\begin{frame}[fragile]
  \frametitle{Typické \texttt{ui.R}}
  \begin{lstlisting}
    library(shiny)
    shinyUI(fluidPage(
      titlePanel("..."),   # název aplikace
      sidebarLayout(
    
        # ovládací prvky aplikace (vstupy; levý panel)
        sidebarPanel(
          #\# první ovládací prvek,
          #\# druhý ovládací prvek,
          #\# ...
        ),
    
        # výstupy; pravý panel
        mainPanel(
          #\# první prvek výstupu,
          #\# druhý prvek výstupu,
          #\# ...
        )
    
      )
    ))
  \end{lstlisting}
\end{frame}


%% server.R -------------------------------------------------------------------

\begin{frame}
  \frametitle{\texttt{server.R}}
  \begin{itemize}
    \item skript s prakticky ryze \textsf{R}-kovým kódem, který obsahuje
    a~definuje všechny aplikací používané funkce a procedury
    \item v podstatě jde o skript, který by měl jít spustit sám o sobě
    v~RStudiu či \textsf{R}-kové konzoli
    \item eventualitou (a lepší) je mít procedury a funkce v separátních
    \texttt{.R} souborech, které bude \texttt{server.R} volat pomocí
    příkazu \texttt{source()} ("modulárně")
  \end{itemize}
\end{frame}


%% typický server.R -----------------------------------------------------------

\begin{frame}[fragile]
  \frametitle{Typický \texttt{server.R}}
  \begin{lstlisting}
    library(shiny)

    shinyServer(function(input, output) {
   
      # kód první funkce/procedury využívající vstupy "z ui.R"
      # a generující výstupy "pro ui.R"
  
      # kód druhé funkce/procedury využívající vstupy "z ui.R"
      # a generující výstupy "pro ui.R"
  
      # ...
  
    })
  \end{lstlisting}
\end{frame}


%% ----------------------------------------------------------------------------
%% sekce "gramatika aplikace" -------------------------------------------------
%% ----------------------------------------------------------------------------

\section{Gramatika aplikace}

%% Gramatika aplikace (ui.R) --------------------------------------------------

\begin{frame}[fragile]
  \frametitle{Gramatika aplikace (\texttt{ui.R})}
  \begin{itemize}
    \item stavebnicový systém
    \item kód pro úroveň vstupů (obvykle levý panel)
    \item \texttt{typ\_ovládacího\_prvku(\ldots)}
  \end{itemize}
  \begin{lstlisting}
    typ_ovládacího_prvku(
      inputId = "id_vstupu",
      argumenty,
      ...
    )
  \end{lstlisting}
  \begin{itemize}
    \item \texttt{typVýstupu(\ldots)}
  \end{itemize}
  \begin{lstlisting}
    typVýstupu(
      outputId = "id_vystupu"
    )
  \end{lstlisting}
\end{frame}


%% Gramatika aplikace (server.R) ----------------------------------------------

\begin{frame}[fragile]
  \frametitle{Gramatika aplikace (\texttt{server.R})}
  \begin{itemize}
    \item taktéž stavebnicový systém
    \item \texttt{vhodnéRenderováníVýstupu(\{\ldots\})}
  \end{itemize}
  \begin{lstlisting}
    output$id_vystupu <- vhodnéRenderováníVýstupu({
  
      # R-ková procedura či funkce beroucí jako vstupy
      input$id_vstupu
  
      # a vracející vhodný výstup
  
    })
  \end{lstlisting}
\end{frame}


%% ----------------------------------------------------------------------------
%% sekce "slovník aplikace" ---------------------------------------------------
%% ----------------------------------------------------------------------------

\section{Slovník aplikace}

%% Slovník aplikace -----------------------------------------------------------

\begin{frame}[fragile]
  \frametitle{Slovník aplikace}
  \begin{itemize}
    \item dvojice správných typů výstupu \texttt{server.R} pro vstupy
    \texttt{ui.R}
  \end{itemize}
  \begin{table}[H]
  \centering
  \begin{tabular}{ll}
    \hline
    \texttt{typVýstupu} & \texttt{vhodnéRenderováníVýstupu} \\
    \hline
    \texttt{verbatimTextOutput(\ldots)} & \texttt{renderPrint(\{\ldots\})} \\
    \texttt{textOutput(\ldots)} & \texttt{renderText(\{\ldots\})} \\
    \texttt{tableOutput(\ldots)} & \texttt{renderTable(\{\ldots\})} \\
    \texttt{plotOutput(\ldots)} & \texttt{renderPlot(\{\ldots\})} \\
    \texttt{uiOutput(\ldots)} & \texttt{renderUI(\{\ldots\})} \\
    \hline
  \end{tabular}
  \end{table}
  \begin{itemize}
    \item kompletní seznam prvků použitelných v \texttt{ui.R}
    a \texttt{server.R} je dostupný zde
    \texttt{ui.R}
  \end{itemize}
  \begin{center}
    \href{https://shiny.rstudio.com/reference/shiny/latest/}{%
      \framebox{https://shiny.rstudio.com/reference/shiny/latest/}%
    }
  \end{center}
\end{frame}


%% ----------------------------------------------------------------------------
%% sekce "co s aplikací" ------------------------------------------------------
%% ----------------------------------------------------------------------------

\section{Co s aplikací}

%% desktopové spuštění --------------------------------------------------------

\begin{frame}
  \frametitle{Desktopové spuštění aplikace}
  \begin{itemize}
    \item lze ji spouštět v RStudiu
    \begin{itemize}
      \item zelený trojúhelníček vpravo nahoře ("Run app")
    \end{itemize}
  \end{itemize}
\end{frame}


%% desktopové spuštění (cont.) ------------------------------------------------

\begin{frame}
  \frametitle{Desktopové spuštění aplikace (cont.)}
  \begin{itemize}
    \item lze ji spustit desktopově bez RStudia
    \begin{itemize}
      \item spouštěcí (\textsf{R}-kový) soubor je na
    \end{itemize}
  \end{itemize}
  \begin{center}
    \href{https://github.com/LStepanek/Vyvoj\_aplikaci\_v\_R/}{%
      \framebox{https://github.com/LStepanek/Vyvoj\_aplikaci\_v\_R/}%
    }
  \end{center}
  \begin{itemize}
    \item nutné spárovat příponu \texttt{.myRscript} s programem, který
    bude aplikaci spouštět, tedy \texttt{spusť\_aplikaci.myRscript}
    \item před prvním spuštěním se po poklepání na tento soubor otevře
    dialog pro nastavení výchozího spouštěcího programu; v lokální nabídce
    přes \textit{Další možnosti} a \textit{Najít jinou aplikaci v tomto
    počítači} vyberme nástroj \texttt{Rscript} ve složce \textit{bin}
    složky \textsf{R}
    \item obvyklá cesta k nástroji vypadá například takto
  \end{itemize}
  \begin{center}
    \framebox{{C:/Program Files/R/R-3.3.0/bin/Rscript}}
  \end{center}
\end{frame}


%% online spuštění ------------------------------------------------------------

\begin{frame}
  \frametitle{Online vystavení aplikace}
  \begin{itemize}
    \item aplikaci lze vystavit zdarma online na web
  \end{itemize}
  \begin{center}
    \href{https://www.shinyapps.io/}{\framebox{https://www.shinyapps.io/}}
  \end{center}
  \begin{itemize}
    \item nutná registrace, zdarma může běžet maximálně pět aplikací
    maximálně 25 hodin měsíčně
    \item link aplikace je obvykle neatraktivní, obsahuje
    doménu \texttt{shinyapps.io}
  \end{itemize}
\end{frame}


%% vlastní R-kový server ------------------------------------------------------

\begin{frame}
  \frametitle{Vlastní \textsf{R}-kový server}
  \begin{itemize}
    \item možné nainstalovat na vlastní web pomocí návodu
  \end{itemize}
  \begin{center}
    \href{https://www.rstudio.com/products/rstudio/download-server/}{%
      \framebox{https://www.rstudio.com/products/rstudio/download-server/}%
    }
  \end{center}
  \begin{itemize}
    \item je třeba se orientovat v Linuxu
    \item je vhodné používat běžnou linuxovou distribuci
    (např. Ubuntu, Fedora, "Debian")
    \item nutné mít možnost root SSH přístupu k serveru,
    což běžný provider neumožní
    \item anebo je možné mít virtuální server na Amazon Web Services,
    více zde
  \end{itemize}
  \begin{center}
    \href{https://aws.amazon.com/blogs/big-data/running-r-on-aws/}{%
      \framebox{https://aws.amazon.com/blogs/big-data/running-r-on-aws/}%
    }
  \end{center}
\end{frame}


%% ----------------------------------------------------------------------------
%% sekce "Další možnosti aplikace" --------------------------------------------
%% ----------------------------------------------------------------------------

\section{Další možnosti aplikace}


%% počítadlo ------------------------------------------------------------------

\begin{frame}[fragile]
  \frametitle{Počítadlo návštěv}
  \begin{itemize}
    \item založeno na permanentním souboru \texttt{counter.RData}
    obsahující dosavadní počet návštěv (jako \texttt{integer})
    \item kód v \texttt{server.R} pak vypadá
  \end{itemize}
  \begin{lstlisting}
    output$counter <- renderText({
      if (!file.exists("counter.Rdata")){
        counter <- 0
      }else{
        load(file = "counter.Rdata")
      }
    
      counter <- counter + 1
  
      save(counter, file = "counter.Rdata")
      paste("Hits:", counter, sep = "")
  
    })
  \end{lstlisting}
\end{frame}


%% busy indikátor -------------------------------------------------------------

\begin{frame}[fragile]
  \frametitle{Busy indikátor}
  \begin{itemize}
    \item založen na třídě \texttt{style.css}, javascriptové funkci
    \texttt{busy.js} (obě ve složce \texttt{www}) a GIFu
    \texttt{free\_busy\_indicator.gif}
    \item kód v \texttt{ui.R} pak vypadá
  \end{itemize}
  \begin{lstlisting}
    tags$head(
      tags$link(rel = "stylesheet",
                type = "text/css",
                href = "style.css"),
      tags$script(type = "text/javascript",
                  src = "busy.js")
    
    ),
    div(class = "busy",
        p("Application is busy..."),
        img(src = "free_busy_indicator.gif", height = 50, width = 50)
    )
  \end{lstlisting}
\end{frame}


%% předchozí prvky-------------------------------------------------------------

\begin{frame}
  \frametitle{Kde získat templáty aditivních funkcionalit}
  \begin{itemize}
    \item sledovat přední vývojáře \textsf{R}-kových aplikací
    (jmenovitě \href{http://deanattali.com/}{Dean Attali})
    \item sledovat jejich GitHubí účty a streamy z use\textsf{R}! konferencí
    \item některé dostupné na
  \end{itemize}
  \begin{center}
    \href{https://github.com/LStepanek/Vyvoj\_aplikaci\_v\_R/}{%
      \framebox{https://github.com/LStepanek/Vyvoj\_aplikaci\_v\_R/}%
    }
  \begin{itemize}
    \item a různě na
  \end{itemize}
    \href{https://github.com/LStepanek/}{%
      \framebox{https://github.com/LStepanek/}%
    }
  \end{center}
\end{frame}


%% ----------------------------------------------------------------------------
%% sekce "příklady aplikací" --------------------------------------------------
%% ----------------------------------------------------------------------------

\section{Příklady aplikací}

%% nastavuji titulní stranu ---------------------------------------------------

\begin{frame}
  \frametitle{Příklady existujících aplikací}
  \begin{itemize}
    \item aplikace \texttt{statisticke\_natroje}
  \end{itemize}
  \begin{center}
    \href{http://shiny.statest.cz:3838/statisticke\_nastroje/}{%
      \framebox{http://shiny.statest.cz:3838/statisticke\_nastroje/}%
    }
  \end{center}
  \begin{itemize}
    \item aplikace \texttt{Conwayova\_hra\_zivota}
  \end{itemize}
  \begin{center}
    \href{http://shiny.statest.cz:3838/Conwayova\_hra\_zivota/}{%
      \framebox{http://shiny.statest.cz:3838/Conwayova\_hra\_zivota/}%
    }
  \end{center}
  \begin{itemize}
    \item aplikace \texttt{lightsout} (Dean Atalli)
  \end{itemize}
  \begin{center}
    \href{http://shiny.statest.cz:3838/lightsout/}{%
      \framebox{http://shiny.statest.cz:3838/lightsout/}%
    }
  \end{center}
  \begin{itemize}
    \item aplikace \texttt{ShinyItemAnalysis} (Patrícia Martinková et al.)
  \end{itemize}
  \begin{center}
    \href{http://shiny.statest.cz:3838/ShinyItemAnalysis/}{%
      \framebox{http://shiny.statest.cz:3838/ShinyItemAnalysis/}%
    }
  \end{center}
\end{frame}


%% ----------------------------------------------------------------------------
%% sekce "konec" --------------------------------------------------------------
%% ----------------------------------------------------------------------------

\section{Konec}

%% poděkování za pozornost ----------------------------------------------------

{
\setbeamertemplate{headline}{} 
\begin{frame}
\frametitle{}
  \vspace{+2.5cm}
  \begin{block}{\centering Děkuji za pozornost!}
    \center
    \begin{tabular}{l@{}l@{}l}
      &\href{mailto:lubomir.stepanek@fbmi.cvut.cz}{%
        lubomir.stepanek@fbmi.cvut.cz%
      } \\
      &\href{mailto:lubomir.stepanek@lf1.cuni.cz}{%
        lubomir.stepanek@lf1.cuni.cz%
      }
    \end{tabular}
  \end{block}
\end{frame}
}


%% ----------------------------------------------------------------------------
%% končím dokument ------------------------------------------------------------
%% ----------------------------------------------------------------------------

\end{document}


%%%%%%%%%%%%%%%%%%%%%%%%%%%%%%%%%%%%%%%%%%%%%%%%%%%%%%%%%%%%%%%%%%%%%%%%%%%%%%%
%%%%%%%%%%%%%%%%%%%%%%%%%%%%%%%%%%%%%%%%%%%%%%%%%%%%%%%%%%%%%%%%%%%%%%%%%%%%%%%
%%%%%%%%%%%%%%%%%%%%%%%%%%%%%%%%%%%%%%%%%%%%%%%%%%%%%%%%%%%%%%%%%%%%%%%%%%%%%%%





